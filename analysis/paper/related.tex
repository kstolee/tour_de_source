\section{Motivation and Related Work}
\label{sec:motivation}

With regexes, there is a common saying: 'now you have two problems'.
%\footnote{\url{http://regex.info/blog/2006-09-15/247}}
A skilled programmer can quickly solve many problems using regular expressions, but these regular expressions can be hard to understand and maintain, resulting in tens of thousands of bug reports~\cite{Spishak:2012:TSR:2318202.2318207}.
%per year???
Regular expression languages enable an irreplaceable search technique used within all kinds of text editors, command line tools and system tools.
Regexes are also employed in critical missions like mysql injection prevention, malicious packet filtering  and web form validation.

Tools like Hampi, Kudzu, brics\footnote{\url{http://www.brics.dk/automaton/}}, Microsoft's Rex, Automata and the z3-based QF\_VRE projects, all attempt to support modeling some subset of regular expression language features, empowering users to do reasoning and validation on regexes.

To improve test coverage for code using regular expressions, and to generate strings from regular expressions for whatever other reasons, projects like Reggae, Rex, JST, regex-tester\footnote{\url{https://github.com/nickawatts/regex-tester}}, regldg\footnote{\url{http://regldg.com/}},
uttool\footnote{\url{http://uttool.com/text/regexstr/default.aspx}}, xeger\footnote{\url{https://code.google.com/p/xeger/}}, Generex\footnote{\url{https://github.com/mifmif/Generex}}, Hoa/Regex\footnote{\url{https://github.com/hoaproject/Regex/}}, Genex\footnote{\url{http://search.cpan.org/~bowmanbs/Regexp-Genex-0.07/lib/Regexp/Genex.pm}}, Randexp\footnote{\url{https://www.ruby-toolbox.com/projects/randexp}}, txt2re\footnote{\url{http://txt2re.com/}}, Pex\footnote{\url{http://research.microsoft.com/en-us/projects/pex/}} (and countless others) have been developed.

Going the other direction, a few projects have attempted to take a set of strings and generate a good regular expression that matches them like RegexGenerator++\footnote{\url{http://regex.inginf.units.it/}}, Regex-PreSuf\footnote{\url{http://search.cpan.org/~jhi/Regex-PreSuf-1.17/PreSuf.pm}} (a problem that suffers from overmatching).
%Tools to help developers write correct regular expressions, tools like RegexBuddy, Expresso, Pythex, The Regex Coach, Regex Widget, Regex magic, RegE xr, reWork, Rubular, Txt2re and....

One common misconception is that all regular expression languages can be represented using deterministic finite automata (DFA), and so they are easy to model, easy to describe formally and execute in O(n) time.  In fact, most regular expression matching engines run in exponential time\footnote{\url{https://swtch.com/~rsc/regexp/regexp1.html}} in order to support useful features such as lazy quantifiers, capturing groups, look-aheads and back-references\footnote{\url{https://msdn.microsoft.com/en-us/library/0yzc2yb0.aspx}}.  In the RE2\footnote{\url{https://github.com/google/re2}} project, Russ Cox aimed to use DFAs as much as possible while supporting as many useful features as possible.

Countless research papers have focused on various other regular expression-related investigations, too many to list.  Because regular expression languages vary somewhat in their syntax and feature set, these papers have had to describe a particular language to reason about and have had to pick what features to include or exclude.

In all of these regex-related projects, researchers and tool designers face a difficult design decision: supporting advanced features is always more expensive, taking more time and potentially making the tool or research project too complex and cumbersome to execute well.  A selection of only the simplest of regex features is common in research papers and automata libraries, but this limits the applicability/relevance of that work in the real world.

The authors of this paper have their own Regex-related tool that they want to implement, and when faced with the inevitable question of what features are okay to exclude, they searched for some empirical research into how regular expressions are used in practice.  Finding no such research that could inform this choice, they decided to do that research themselves and hopefully empower other researchers with that information.




\subsection{Research on Regular Expressions}
Visual debugging of regular expressions~\cite{Beck:2014:RVD:2591062.2591111}

%the related work section in the Spishak section is very good re: regex tools like those that represent regexes as automata or grammars
Static analysis to reduce errors in building regular expressions by using a type system to identify errors like {\tt PatternSyntaxExceptions} and {\tt IndexOutOfBoundsExceptions} at compile time~\cite{Spishak:2012:TSR:2318202.2318207}.

\subsection{Research on Regular Expressions}
Visual debugging of regular expressions~\cite{Beck:2014:RVD:2591062.2591111}

\subsection{Research that Depends on Regular Expression Usage}
Regular expressions are used as queries in a data mining framework~\cite{Begel:2010:CDE:1806799.1806821}

