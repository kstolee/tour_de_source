\section{Related Work}
\label{sec:related}

%
%Going the other direction, a few projects have attempted to take a set of strings and generate a good regular expression that matches them like RegexGenerator++\footnote{\url{http://regex.inginf.units.it/}} and Regex-PreSuf\footnote{\url{http://search.cpan.org/~jhi/Regex-PreSuf-1.17/PreSuf.pm}} (a problem that suffers from overmatching).
%%Tools to help developers write correct regular expressions, tools like RegexBuddy, Expresso, Pythex, The Regex Coach, Regex Widget, Regex magic, RegE xr, reWork, Rubular, Txt2re and....

Regular expressions have been a focus point in a variety of research objectives. From the user perspective, tools have been developed to support more robust creation~\cite{Spishak:2012:TSR:2318202.2318207} or to allow visual debugging~\cite{Beck:2014:RVD:2591062.2591111}.
Building on the perspective that regexes are difficult to create, other research has focused on removing the human from the creation process by learning regular expressions from  text~\cite{Babbar:2010:CBA:1871840.1871848, Li:2008:REL:1613715.1613719}.

Regarding applications, regular expressions have been used for test case generation~\cite{Ghosh:2013:JAT:2486788.2486925, Galler:2014:STD:2683035.2683100, Anand:2013:OSM:2503903.2503991, Tillmann:2014:TAT:2642937.2642941},  and
solvers for string constraints~\cite{Trinh:2014:SSS:2660267.2660372, hampi}.
Regexes are also employed in critical missions like mysql injection prevention~\cite{Yeole:2011:ADT:1980022.1980229} and network intrusion detection~\cite{network}, or in more diverse applications like DNA sequencing alignment~\cite{1594922} or querying RDF data~\cite{Lee:2010:PSQ:1871871.1871877, Alkhateeb:2009:ESR:1540656.1540975}.


As a query language, lightweight regular expressions are pervasive in search. For example,
some data mining frameworks use regular expressions as queries (e.g., ~\cite{Begel:2010:CDE:1806799.1806821}). Efforts have also been made to expedite the processing of regular expressions on large bodies of text~\cite{Baeza-Yates:1996:FTS:235809.235810}. 

%One common misconception is that all regular expression languages are \emph{regular languages} which can be represented using deterministic finite automata (DFA), and so they are easy to model, easy to describe formally and execute in O(n) time.  In fact, many regular expression matching engines run in exponential time in order to support useful features such as lazy quantifiers, capturing groups, look-aheads and back-references~\cite{msdnmatching}.  In a recent regular expression library, the RE2 projext~\cite{re2}, Russ Cox aimed to use DFAs as much as possible (maximizing speed) while supporting as many useful features as possible.

%Thousands of research papers have focused on various other regular expression-related investigations.
Within standard programming languages, regular expressions libraries are very common, yet there are  differences between languages in the features that they support. For example, Java supports possessive quantifiers like \verb! `ab*+c'! (here the `+' is modifying the `*' to make it possessive) whereas Python does not.

Since  regular expression languages vary somewhat in their syntax and feature set, researchers and tool designers have typically had to pick what features to include or exclude. Thus, researchers and tool designers face a difficult design decision: supporting advanced features is always more expensive, taking more time and potentially making the tool or research project too complex and cumbersome to execute well.  A selection of only the simplest of regex features is common in research papers and automata libraries, but this limits the applicability/relevance of that work in the real world.



In this work, we perform a feature analysis on regular expressions used in the wild and compare that set to the features supported by four popular regular expression tools.
Research tools like Hampi~\cite{hampi}, and Rex~\cite{rex}, and commercial tools like brics\cite{brics} and RE2~\cite{re2}, all use regular expressions for various task. Hampi was developed  in academia and uses regular expressions as a specification language for a strong constraint solver. Rex was developed by Microsoft Research and generates strings for regular expressions that can be used in several applications, such as test case generation~\cite{Anand:2013:OSM:2503903.2503991, Tillmann:2014:TAT:2642937.2642941}. Brics is an open-source package that creates automata from regular expressions for manipulation and evaluation.
RE2 is an open-source tool created by Google to power Code Search with a more efficient regex engine.
While there are many regular expression tools available, in this work, we focus on the features support for these four tools, which offer diversity across developers (i.e., Microsoft, Google, open source, and academia) and across applications. Further, as the focus of this work is on tool designers and we wanted to perform a feature analysis, these four tools and their features are well-documented, allowing for easy comparison.

Mining properties of open source repositories is a well-studied topic, focusing, for example, on API usage patterms~\cite{Linares-Vasquez:2014:MEA:2597073.2597085} and bug characterizations~\cite{Chen:2014:ESD:2597073.2597108}.
To our knowledge, this is the first work to mine and evaluate regular expression usages from existing software repositories. Related to mining work, regular expressions have been used to form queries in mining framework~\cite{Begel:2010:CDE:1806799.1806821}, but have not been the focus of the mining activities.

% \subsection{Research on Regular Expressions}
% Visual debugging of regular expressions~\cite{Beck:2014:RVD:2591062.2591111}

% %the related work section in the Spishak section is very good re: regex tools like those that represent regexes as automata or grammars
% Static analysis to reduce errors in building regular expressions by using a type system to identify errors like {\tt PatternSyntaxExceptions} and {\tt IndexOutOfBoundsExceptions} at compile time~\cite{Spishak:2012:TSR:2318202.2318207}.

% \subsection{Research on Regular Expressions}
% Visual debugging of regular expressions~\cite{Beck:2014:RVD:2591062.2591111}

% \subsection{Research that Depends on Regular Expression Usage}
% Regular expressions are used as queries in a data mining framework~\cite{Begel:2010:CDE:1806799.1806821}

