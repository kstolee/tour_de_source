\section{Survey}
\label{sec:survey}

To understand the context of when and how programmers use regular expressions,
we designed a survey with 41 questions about regex usage. This survey was
deployed to developers at Dwolla, a company that provides software for
 online and mobile payment management.
 \todoLast{they are probably ok with using their name - check during final scrub by Dwolla}
Participation was voluntary and participants were entered in a lottery for a \$50 gift card.
The survey was completed by 18 participants that identified as software developer/maintainers who had used regular expressions in a work environment.


On average, survey participants report to compose 172 regexes per year ($\sigma$ = 250) and compose regexes on average once per month, with 27\% composing multiple regexes in a week and an additional 22\% composing regexes once per week.
Table~\ref{tab:regexenviron} shows how frequently participants compose regexes using each of several languages and technical environments.
Six (33\%) of the survey participants report to compose regexes using general purpose programming languages (e.g., Java, C, C\#) 1-5 times per year and 5 (28\%) do this 6-10 times per year.  Regexes were rarely used in query languages like SQL, but for command line usage in tools such as grep, 6 (33\%) participants use regexes 51+ times per year. Overall, regexes are used frequently, but in some environments, such as command line or text editor, and sometimes query languages, the composed regular expressions do not persist.

\newcommand{\horiz}{\hspace{2.1pt}}

\begin{table}
\caption{Survey results for number of regexes composed per year by technical environment \label{tab:regexenviron}}
\begin{center}
\begin{small}
\begin{tabular}{l | r @{  \horiz} r @{ \horiz } r @{ \horiz } r @{ \horiz } r @{ \horiz } r }
Language/Environment & 0 & 1-5 & 6-10 & 11-20 & 21-50 & 51+ \\ \hline
General  (e.g., Java)  & 1 & 6 & 5 & 3& 1& 2 \\
Scripting  (e.g., Perl) &5 &4 &3 &3 &2  &1 \\
Query  (e.g., SQL) & 15&2 &0 &0 &1  & 0\\
Command line (e.g., grep)   &2 &5 &3 &2 &0  &6 \\
Text editor (e.g., IntelliJ)   & 2& 5& 0& 5& 1& 5\\
\end{tabular}
\end{small}
\end{center}
\end{table}

\begin{table}
\caption{Survey results for regex usage frequencies for various activities, averaged using a 6-point likert scale: Very Frequently=6, Frequently=5, Occasionally=4, Rarely=3, Very Rarely=2, and Never=1 \label{tab:regexactivities}}
\begin{center}
\begin{tabular}{l|c}
Activity & Frequency \\ \hline
Locating content within a file or files & 4.4\\
Capturing parts of strings & 4.3 \\
Parsing user input & 4.0\\
Counting lines that match a pattern & 3.2\\
Counting  substrings that match a pattern & 3.2\\
Parsing generated text & 3.0\\
Filtering collections (lists, tables, etc.) & 3.0 \\
Checking for a single character & 1.7\\


\end{tabular}
\end{center}
\end{table}

Table~\ref{tab:regexactivities} shows how frequently, on average, the participants use
regexes for various actives.
Participants answered questions using a 6-point likert scale including very frequently, frequently, occasionally, rarely, very rarely, and never.
Assigning values from 1 to 6, where 6 is the most frequent, the responses were averaged across participants.
\todoLast{is the 6-point scale overemphasized here and in the table caption?}
Among the most common are capturing parts of a string and locating content within a file, with both occurring somewhere between occasionally and frequently.

Using a similar 7-point likert scale that includes 'always' as a seventh point, developers indicated that they test their code with the same frequency as they test their regexes (5.2 which is between frequently and very frequently).  Half of the 18 developers indicate that they use tools to test their regexes, and the other half indicated that they only use tests that they write themselves. Of the 9 developers using tools, 6 of them mentioned some online composition aide similar to regex101.com where a regex and input are entered, and the input is highlighted according to what is matched.
%The other three developers mentioned 'ScalaCheck' (a testing framework), 'IDE regex plugins', and 'Language specific Regexlib'.

When asked an open ended question about pain points encountered with regular expressions, 7 developers responded with an answer equivalent to 'hard to read', 3 responded with 'inconsistency across implementations' and 11 responded with 'hard to compose' (3 developers gave two answers).

The survey validates the assumption that regex are widely used by professional software developers.  Future research into regex can focus on areas that prove to be most important to developers, namely capturing parts of strings and searching for specific content.  Although research into regex use in general purpose and scripting languages is important, usage within command line tools and text editors should also be considered.  The fact that all the surveyed developers compose regexes, and half of the developers use tools to test their regexes indicates the importance of tool development for regex.  Developers complain about regex being hard to read and hard to compose, and most of the tools that they indicate using are focused on composition, indicating a need for tools that help make existing regexes more readable.
