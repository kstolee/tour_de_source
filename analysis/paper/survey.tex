\section{Survey}
\label{sec:survey}

To understand the context of when and how programmers use regular expressions,
we designed a survey with 41 questions about regex usage. This survey was
deployed to 22 developers at Dwolla, a company that provides software for
 online and mobile payment management.
 \todoLast{they are probably ok with using their name - check during final scrub by Dwolla}
Participation was voluntary and participants were entered in a lottery for a \$50 gift card.
The survey was completed by 18 participants (82\% response rate) that identified as software developer/maintainers. The questions asked about general regex usage and the use of various language features. 


\subsection{General Regex Usage}
On average, survey participants report to compose 172 regexes per year ($\sigma$ = 250) and compose regexes on average once per month, with 28\% composing multiple regexes in a week and an additional 22\% composing regexes once per week. That is, 50\% of respondents uses regexes at least weekly. 
Table~\ref{tab:regexenviron} shows how frequently participants compose regexes using each of several languages and technical environments.
Six (33\%) of the survey participants report to compose regexes using general purpose programming languages (e.g., Java, C, C\#) 1-5 times per year and five (28\%) do this 6-10 times per year.  Regexes were rarely used in query languages like SQL, but for command line usage in tools such as grep, 6 (33\%) participants use regexes 51+ times per year. Overall, regexes are used frequently, but in some environments, such as command line or text editor, and sometimes query languages, the composed regular expressions do not persist.

\newcommand{\horiz}{\hspace{2.1pt}}

\begin{table}
\caption{Survey results for number of regexes composed per year by technical environment \label{tab:regexenviron}}
\begin{center}
\begin{small}
\begin{tabular}{l | r @{  \horiz} r @{ \horiz } r @{ \horiz } r @{ \horiz } r @{ \horiz } r }
Language/Environment & 0 & 1-5 & 6-10 & 11-20 & 21-50 & 51+ \\ \hline
General  (e.g., Java)  & 1 & 6 & 5 & 3& 1& 2 \\
Scripting  (e.g., Perl) &5 &4 &3 &3 &2  &1 \\
Query  (e.g., SQL) & 15&2 &0 &0 &1  & 0\\
Command line (e.g., grep)   &2 &5 &3 &2 &0  &6 \\
Text editor (e.g., IntelliJ)   & 2& 5& 0& 5& 1& 5\\
\end{tabular}
\end{small}
\end{center}
\end{table}

\begin{table}
\caption{Survey results for regex usage frequencies for various activities, averaged using a 6-point likert scale: Very Frequently=6, Frequently=5, Occasionally=4, Rarely=3, Very Rarely=2, and Never=1 \label{tab:regexactivities}}
\begin{center}
\begin{tabular}{l|c}
Activity & Frequency \\ \hline
Locating content within a file or files & 4.4\\
Capturing parts of strings & 4.3 \\
Parsing user input & 4.0\\
Counting lines that match a pattern & 3.2\\
Counting  substrings that match a pattern & 3.2\\
Parsing generated text & 3.0\\
Filtering collections (lists, tables, etc.) & 3.0 \\
Checking for a single character & 1.7\\


\end{tabular}
\end{center}
\end{table}

Table~\ref{tab:regexactivities} shows how frequently, on average, the participants use
regexes for various actives.
Participants answered questions using a 6-point likert scale including very frequently, frequently, occasionally, rarely, very rarely, and never.
Assigning values from 1 to 6, where 6 is the most frequent, the responses were averaged across participants.
Among the most common usages are capturing parts of a string and locating content within a file, with both occurring somewhere between occasionally and frequently.

Using a similar 7-point likert scale that includes `always' as a seventh point, developers indicated that they test their regexes with the same frequency as they test their code (average response was 5.2, which is between frequently and very frequently).  Half of the 18 developers indicate that they use external tools to test their regexes, and the other half indicated that they only use tests that they write themselves. Of the nine developers using tools, six mentioned some online composition aide such as \url{regex101.com} where a regex and input string are entered, and the input string is highlighted according to what is matched.
%The other three developers mentioned 'ScalaCheck' (a testing framework), 'IDE regex plugins', and 'Language specific Regexlib'.

When asked an open ended question about pain points encountered with regular expressions, we observed three main categories. The most common, ``hard to compose," was represented in 61\% (11) responses. Next, 
 39\% (7) developers responded that regexes are ``hard to read" and 17\% (3) indicated difficulties with ``inconsistency across implementations," which manifest when using regexes in multiple languages. These responses do not sum to 18 as three developers provided overlapping  answers.

\subsection{Regex Feature Usage}
The pattern language for Python, which is used to compose regexes, supports default character classes like the ANY or dot character class: \verb!.! meaning, `any character except newline' (a full list of features and examples is in the first four columns of Table~\ref{table:featureStats}).  
It also supports three other default character classes: \verb!\d!, \verb!\w!, \verb!\s! (and their negations). All of these default character classes can be simulated using the custom character class (CCC) feature, which can create semantically equivalent regexes.  
For example  the decimal character class: \verb!\d! is equivalent to a CCC containing all 10 digits:  \verb!d! $\equiv$ \verb![0123456789]! $\equiv$ \verb![0-9]!.  
%Users have a choice when using regex between writing \verb!\d! and \verb![0-9]! and whereas the first option may be shorter, the second may seem more intuitive and readable.  
Other default character classes such as the word character class: \verb!\w! may not be as intuitive to encode in a CCC: \verb![a-zA-Z0-9_]!.  

Survey participants were asked if they use only CCC, use CCC more than default, use both equally, use default more than CCC or use only default.  Results for this question are shown in Table~\ref{tab:cccvsdefault}, with 67\% (12) indicating that they use default the most. 
Participants were also asked to explain their preferences.  Participants who favored CCC mostly said something equivalent to ``it is more explicit," whereas the participants who favored default character classes said,  ``it is less verbose" and ``I like using built-in code."

\begin{table}
\caption{Survey results for preferences between custom character and default character classes. \label{tab:cccvsdefault}}
\begin{center}
\begin{tabular}{l|c}
Preference & Frequency \\ \hline
use only CCC & 1\\
use CCC more than default & 5 \\
use both equally & 2\\
use default more than CCC & 10\\
use only default & 2\\

\end{tabular}
\end{center}
\end{table}

To further explore how frequently participants use various regex features, participants were asked five questions (on a 6-point likert scale) about how frequently they use specific related groups of features:
\begin{itemize} \itemsep -2pt
    \item endpoint anchors: \verb!^! and \verb!$!
    \item capture groups: (capture me)
    \item word boundaries: \verb!word\b!
    \item (negative) look-ahead/behinds: \verb!a(?=bc)!, \verb!(?<!x)yz!, \verb!(?<=a)!, \verb!a(?!yz)!
    \item lazy repetition: \verb!ab+?!, \verb!xy{2,3}?!
\end{itemize}
These features were chosen based on \todoMid{why just these features?}
Results are shown in Table~\ref{tab:regexfeaturegroups}, indicating that lazy repetition and look-ahead features are rarely used and capture groups and endpoint anchors are occasionally to frequently used. 

\begin{table}
\caption{Survey results for regex usage frequencies for five feature groups, averaged using a 6-point likert scale: Very Frequently=6, Frequently=5, Occasionally=4, Rarely=3, Very Rarely=2, and Never=1 \label{tab:regexfeaturegroups}}
\begin{center}
\begin{tabular}{l|c}
Group & Frequency \\ \hline
endpoint anchors & 4.4\\
capture groups & 4.2 \\
word boundaries & 3.5 \\
lazy repetition & 2.9\\
(negative) look-ahead/behinds & 2.5\\


\end{tabular}
\end{center}
\end{table}

The survey validates the assumption that regex are widely used by professional software developers and sheds some light into the context in which regexes are used. 
%Future research into regex can focus on activities that prove to be most important to developers, namely capturing parts of strings and searching for specific content.  
%Although research into regex use in general purpose and scripting languages is important, usage within command line tools and text editors should also be considered.  
The fact that all the surveyed developers compose regexes, and half of the developers use tools to test their regexes indicates the importance of tool development for regex.  Developers complain about regex being hard to read and hard to compose, and most of the tools that they indicate using are focused on composition, indicating a need for tools that help make existing regexes more readable.
