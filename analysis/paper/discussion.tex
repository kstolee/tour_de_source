\section{Discussion}
\label{sec:discussion}

\todo{Carl: this section needs a lot of focus and thought ASAP}

\subsection{Implications for Tool Designers}

\subsubsection{character classes are important}
not supporting {\tt \\d} makes a big difference.

\subsubsection{Anchors matter}
The endpoint anchor features STR and END are the only way specify that an entire line has to match.  Consider the following example, comparing the pattern \verb!`^\s*$'! (found in 48 projects) to the pattern \verb!`\s*'! when looking for empty lines in a text file for a diff utility.  Without the endpoint anchors, the pattern matches every line, since there are always at least zero whitespace characters on every line.  But with the endpoint anchors, only lines that contain nothing but whitespace will match.

\subsection{Opportunities for Future Work}

\subsubsection{New library feature for properties of a line}

\subsubsection{Regexes need refactoring}
We see the same features implemented many ways, and we don't know why. It might be that some methods of implementation are more understandable. However, if tool designers do not support that, refactoring may be needed.

example with anchors and {\tt .*} in the middle which could be replaced by a comma?

\subsection{Frequency of {\tt re} Module Usage}
...only \DTLfetch{data}{key}{percentFilesUsingRegex}{value}\% of the files observed had at least one regex usage.  This indicates that regex usage may usually be concentrated in just a few files.


From table~\ref{table:saturation}, we see that on average each project had \DTLfetch{data}{key}{medRFilePerProj}{value} files containing any regex usage, out of an average of \DTLfetch{data}{key}{medFilePerProj}{value} files.  Each of the files that did have a regex usage had an average of \DTLfetch{data}{key}{medRPerFile}{value} regex usages.  Because we scanned \DTLfetch{data}{key}{nProjScanned}{value} projects, we would expect to have seen \DTLfetch{data}{key}{nExpectedUsages}{value} regex usages, which is lower than the actual \DTLfetch{data}{key}{nUsages}{value} usages observed.
\subsubsection{Saturation}Although \DTLfetch{data}{key}{percentProjectsUsingRegex}{value}\% of the projects observed had at least one regex usage, only \DTLfetch{data}{key}{percentFilesUsingRegex}{value}\% of the files observed had at least one regex usage.



Fun fact: while creating similarity matrix, row 5464 took 2 hours, or almost 1 second per cell avg, only suffering 18 timeouts (1.2 secs).  What is this pesky pattern?

We do not assume that Python projects represent a perfect sample of regular expression usage in all environments, but to make the work of collecting data for the paper reasonable, we had to choose one language to focus on (we hope to compare results across languages in future work).  Python is an attractive choice because the culture of Python programming makes it seem likely that someone would write the pattern directly in the function, not trying to over-complicate things with some extra Classes or functions.  Other attractive choices are Perl (which probably has the most active regex community), javascript and ruby (which may emphasize web tasks like form validation), sql or a general purpose language like java or C\#.

Features discussion:

Capture groups are arguably the most powerful feature, since they allow us to get the useful information that we were searching for.  They are probably used simply to group some sequence even when there is no intention of getting the content of the group.

For some reason I want to lump the OR in with CG, or try it.

We can logically group all repetition into the DBB type, since QST={tt \{0,1\}}, SNG={tt \{N,N\}}, and KLE={tt \{0,MAX\}}, ADD={tt \{1,MAX\}}, and LWB={tt \{1,MAX\}} for some MAX like Integer.MAX\_VALUE.

But I want to look at ADD and KLE on their own - or look at them as super conveneint shorthand.  SO convenience needs to enter the discussion, not just equivalence.

We can group STR, END and ENDZ together as endpoint anchors.

We can group PNG, BKR and BKRN together as all back-referencing.

Really, CCC, NCCC and RNG are all very fundamental, and supported by all.  It is worth looking at what they do.

Meanwhile you could say that WSP, DEC, WRD and their opposites are reducible to CCC, but it would be interesting to see how often they are used.

ANY could also be reduced, but I think it should be treated separately from these 6.

I think that looking ahead and looking behind can be separated into two things, though the NCG and LKA - I want to know if they are not equivalent?

WNW boundary and the opposite is special enough to do on its own.

LZY gets its own show and is special.

OPT - I have no idea, look into it.

VWSP - super rare, maybe ignore it.








